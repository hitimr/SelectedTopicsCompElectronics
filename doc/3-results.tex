\section{Results}

Fig. \ref{fig:results} shows the achieved performance for different problem sizes on all three kernels.
NanoVDB achieves overall better results on the CPU compared to OpenVDB. 
Above one million rays the GPU starts to overtake both CPU kernels.

OpenVDB achieves up to 18.5 MRps. NanoVDB consistently outperforms OpenVDB and reaches up to 29.7 MRps.
For problems with 1 million rays or more the GPU kernel overtakes both CPU implementations and achieves up to 117.4 MRps.
Therefore a switch from OpenVDB to NanoVDB on a similar priced GPU can increase performance by a factor of 6.3 or more.
 
Furthermore both CPU implementations suffer from random drops in performance while GPU results are more consistent. 


\begin{figure}[h]
    % Rigth image
    \begin{subfigure}{0.5\textwidth}
    \includegraphics[width=1\linewidth]{res/results.pdf} 
    %\caption{Caption1}
    %\label{fig:serial-solution}
    
\end{subfigure}
    % left image
    \begin{subfigure}{0.4\textwidth}
    \includegraphics[width=1\linewidth]{res/barplot.pdf}
    %\caption{Caption 2}
    %\label{fig:parallel-solution}
\end{subfigure}

\caption{Left: measured performance across different problem sizes. Right: Best results for each kernel}
\label{fig:results}
\end{figure}